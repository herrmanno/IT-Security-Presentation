\documentclass{beamer}
\usepackage[ddmmyyyy]{datetime}
\renewcommand{\dateseparator}{.}
%\usetheme{Boadilla}
% \usetheme{Frankfurt} % like HTWK
\usetheme{Madrid}
\usecolortheme{dolphin}

\title{Google Pixel Lockscreen Bypass}
\subtitle{Race conditions gone wild}
\author{Oliver Herrmann}
\institute{HTWK Leipzig}
\date{\today}

\AtBeginSection[]{
  \begin{frame}
  \vfill
  \centering
  \begin{beamercolorbox}[sep=8pt,center,shadow=true,rounded=true]{title}
    \usebeamerfont{title}\insertsectionhead\par%
  \end{beamercolorbox}
  \vfill
  \end{frame}
}

\setbeamertemplate{itemize items}[circle]
\setbeamertemplate{enumerate items}[square]

\begin{document}

\begin{frame}
\titlepage
\end{frame}

\begin{frame}
\frametitle{Agenda}
\tableofcontents
\end{frame}

\section{Übersicht}

\begin{frame}
\frametitle{\href{https://cve.mitre.org/cgi-bin/cvename.cgi?name=CVE-2022-20465}{CVE-2022-20465}}
\begin{itemize}
    \begin{block}{Beschreibung}
        ``Possible lockscreen bypass due to a logic error in the code''
    \end{block}
    \item Gemeldet: 13.06.2022
    \item Behoben: 05.11.2022
    \item Schere: hoch
    \item Typ: Elevatin of privileges
    \item Betroffen
    \begin{itemize}
        \item Geräte: Google Pixel
        \item Systeme: Android 10 - Android 13
    \end{itemize}
\end{itemize}
\end{frame}

\begin{frame}
\frametitle{Schwachstelle}
\begin{enumerate}
    \item Eingeschaltetes, \textbf{gesperrtes}, Google Pixel Smartphone
    \item Entfernen der SIM-Karte
    \item Einsetzen einer SIM-Karte
    \item Dreimaliges eingeben einer falschen SIM-Karten-PIN
    \item Eingeben der SIM-Karten-PUK
    \item Das Smartphone ist \textbf{entsperrt}
\end{enumerate}
\end{frame}

\begin{frame}
\frametitle{Ursache}
\begin{itemize}
    \item Race-condition beim Freigeben von ``Security-Screens''
    \begin{itemize}
        \item Security-Screens liegen als Stack vor
        \item Freigeben entfernt den ``aktiven'' Security-Screen
    \end{itemize}
    \item SIM-Monitor Prozess setzt Standard-Security-Screen aktiv bei Änderung des SIM-Zustands
    \item PUK-Security-Screen gibt fälschlicher Weise den aktiven Security-Screen (nicht sich selbst) frei
\end{itemize}
\end{frame}

\begin{frame}
\frametitle{Fehlerbehebung}
\begin{block}{Fehlerbehebung}
    Beim Freigeben von Security-Screens muss angegeben werden welche Art
    von Security-Screen freigegeben werden soll.
    
    \hfill \href{https://github.com/aosp-mirror/platform_frameworks_base/commit/ecbed81c3a331f2f0458923cc7e744c85ece96da}{Commit}
\end{block}
\end{frame}

% \begin{frame}
% \frametitle{Ähnliche Fälle}
% \end{frame}

\section{Bedrohungsanalyse (STRIDE)}

\begin{frame}
\frametitle{Spoofing}
\begin{itemize}
    \item Die Sicherheitslücke ist nicht von Spoofing im klassischen Sinne betroffen
    \item Bei Ausnutzung der Lücke kann eine fremde Identität ggü. einer dritten Partei vorgegeben werden
    \item Hierfür ist aber \textbf{kein Verändern von Daten} (in einem Netzwerk) nötig
\end{itemize}
\end{frame}

\begin{frame}
\frametitle{Tampering}
\begin{itemize}
    \item Die Sicherheitslücke ist von Tampering betroffen
    \item Bei Ausnutzung der Lücke kann der Zustand / die Daten des betroffenen
    Geräts ohne Kenntnis der Inhabers verändert werden
\end{itemize}
\end{frame}

\begin{frame}
\frametitle{Repudiation}
\begin{itemize}
    \item Die Sicherheitslücke ist von Repudiation betroffen
    \item Bei Ausnutzung der Lücke können mit dem betroffenen Gerät Prozesse
    ausgeführt werden, die der Geräteinhaber nachträglich nicht zweifelsfrei
    abstreiten kann
\end{itemize}
\end{frame}

\begin{frame}
\frametitle{Information Disclosure}
\begin{itemize}
    \item Die Sicherheitslücke ist von Information Disclosure betroffen
    \item Bei Ausnutzung der Lücke können Informationen, die auf dem Gerät
    prinzipiell geschützt vorliegen, eingesehen werden
\end{itemize}
\end{frame}

\begin{frame}
\frametitle{Denial of Service}
\begin{itemize}
    \item Die Sicherheitslücke ist nicht direkt von Denial of Service betroffen
    \item Bei Ausnutzung der Lücke ist denkbar, dass der Angreifer das Gerät so
    manipuliert, dass es nicht mehr nutzbar ist
    \item Hierfür sind aber ggf. weitere Sicherheitslücken notwendig
\end{itemize}
\end{frame}

\begin{frame}
\frametitle{Elevation of Privilege}
\begin{itemize}
    \item Die Sicherheitslücke ist von Elevation of Privilege betroffen
    \item Bei Ausnutzung der Lücke kann ein Angreifer das Gerät prinzipiell mit
    den Rechten des Gerätebesitzers bedienen
\end{itemize}
\end{frame}

\section{Fazit}

\begin{frame}
\frametitle{Fazit}
\begin{itemize}
    \item Die Sicherheitslücke gehört in die Kategorie ``worst case'' 
    \begin{itemize}
        \item Bei Ausnutzung sind viele Arten von Schäden möglich
        \item Die Lücke kann ohne besondere technische Erfordernisse / Know-how ausgenutzt werden
        \item Das Ausnutzen der Lücke erfordert wenig Zeit / Ressourcen
    \end{itemize}
    \item Es ist physischer Zugriff aus das Gerät notwendig
\end{itemize}
\end{frame}

\begin{frame}
\frametitle{Quellen}
\begin{itemize}
    \fontsize{8pt}{8pt}\selectfont
    \item Blog: \url{https://bugs.xdavidhu.me/google/2022/11/10/accidental-70k-google-pixel-lock-screen-bypass/}
    \item CVE: \url{https://cve.mitre.org/cgi-bin/cvename.cgi?name=CVE-2022-20465}
    \item Android Security Bulletin: \url{https://source.android.com/docs/security/bulletin/2022-11-01}
\end{itemize}
\end{frame}

\end{document}
